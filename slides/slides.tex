\documentclass[style=mindhack]{powerdot}
\usepackage[T1]{fontenc}
\usepackage{textcomp}
\usepackage{pst-node}
\usepackage{tabularx}
\usepackage{calc}
\usepackage{multido}
\usepackage{units}

\renewcommand{\thefootnote}{\fnsymbol{footnote}}

\newcommand\hypS{\pscirclebox[linecolor=blue!40!black,framesep=.5pt]{\textcolor{blue!40!black}{\textbf{S}}}}
\newcommand\hypH{\psframebox[linecolor=green!60!black,framesep=1.5pt]{\textcolor{green!60!black}{\textbf{H}}}}

\title{Hypnosis 101}
\author{\textbf{Dave Ingram}\vspace*{3pt}\\\footnotesize with help from\vspace*{3pt}\\\small Matt Peperell and Morris Webb}
%\date{4th November 2010}

\hypersetup{
	pdfauthor={Dave Ingram},
	pdftitle={Hypnosis 101},
	pdfsubject={Hypnosis 101},
pdfkeywords={Hypnosis},
	pdfview={FitH},
	pdfpagelayout={SinglePage}
}

\DeclareTextCommand{\_}{OT1}{%
  \leavevmode \kern.06em\vbox{\hrule width.5em}}

\newcommand\hl[2]{{\onslide*{#1}{\bfseries\color{red}}#2}}

\begin{document}

	\maketitle

  \begin{slide}{Hypnosis process}%
    \vspace*{\stretch{1}}
    \begin{center}
      \begin{minipage}{.35\linewidth}
        \Large
        \begin{enumerate}
          \item Pre-talk
          \item Testing
          \item \hl{2}{Induction}
          \item \hl{2}{Deepening}
          \item \hl{2}{Suggestion}
          \item \hl{2}{Awakening}
          \item \hl{2}{Posthypnotic}
          \item Debrief
        \end{enumerate}
      \end{minipage}
    \end{center}
    \vspace*{\stretch{1}}
  \end{slide}

  \section{Inductions}

  \begin{slide}{Narrative (staircase)}
    \textbf{Basic idea:} \hypH\ talks \hypS\ through a story, such as
    descending a staircase, introducing suggestions of relaxation.

    \begin{enumerate}
      \item \hypH\ explains the idea to \hypS\ and asks them to close their eyes to start
      \item ``Imagine yourself at the top of a staircase, stretching away in front of you.''
      \item ``I'm going to count from 10 down to 1, and I want you to imagine heading down that staircase. When I reach 1 you'll be at the bottom.''
      \item \hypH\ talks the person down through the numbers, giving suggestions of slow relaxation across the body
      \item Perhaps start with the neck, then shoulders, arms, torso, legs
      \item ``Heavy'', ``relaxed'', ``loose'', ``melting'', all good concepts to use
    \end{enumerate}
  \end{slide}

  \begin{slide}{Eye fixation}
    \begin{enumerate}
      \item \hypH\ instructs \hypS\ to focus their eyes on a point just above their eyeline
      \item \hypH\ is going to count backwards from 100, with \hypS\ opening eyes on even numbers and closing on odd
      \item \hypH\ tells \hypS\ to take a deep breath or two, then starts the slow countdown
      \item \hypH\ starts to introduce suggestions of relaxation and eyelid heaviness while pausing longer with eyes closed
      \item \hypH\ can introduce confusion by missing out later numbers occasionally
    \end{enumerate}
  \end{slide}

  \begin{slide}{Clock face}%
    \begin{enumerate}
      \item \hypS\ holds imaginary clock at arms' length, just above their eyes
      \item \hypS\ imagines the clock pointing at 1 and \hypH\ says ``One, relaxing, going deeper''
      \item \hypS\ imagines the clock pointing at 2 and \hypH\ says ``Two, two, relaxing, relaxing, going deeper, going deeper''
      \item \ldots and continue, repeating each part the appropriate number of times
      \item \hypH\ can make suggestions of relaxation and eye closure between numbers
    \end{enumerate}
  \end{slide}

\end{document}

% vim: et ts=2 sw=2
